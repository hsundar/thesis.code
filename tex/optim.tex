\def\singlesidestandardsetup{

\oddsidemargin .0 in

\topmargin -0.15 in

\hoffset -0.05 in

\textheight 9 in

\baselineskip 5 in

\textwidth 6.5 in}


\documentclass[10pt]{article}

\pagestyle{plain}

\singlesidestandardsetup

\renewcommand{\baselinestretch}{1.0}

\usepackage{amsmath,amssymb,enumerate,bbm}

%%%%%%%%%% Start TeXmacs macros
\newcommand{\tmstrong}[1]{\textbf{#1}}
\newcommand{\tmmathbf}[1]{\ensuremath{\boldsymbol{#1}}}
\newcommand{\tmop}[1]{\ensuremath{\operatorname{#1}}}
\newtheorem{notation}{Notation}
\newenvironment{enumerateroman}{\begin{enumerate}[i.]}{\end{enumerate}}
%%%%%%%%%% End TeXmacs macros
\newfont{\amsbold}{msbm10}
\newfont{\logobold}{logobf10 scaled\magstep2}
\newcommand{\real}{\mathbb{R}}
\newcommand{\assembly}{\mathop{\mbox{\logobold A}}}
\def\nsd{{n_\mathrm{sd}}}
\def\isd{{i_\mathrm{sd}}}
\def\nen{{n_\mathrm{en}}}
\def\nel{{n_\mathrm{el}}}
\def\iel{{i_\mathrm{el}}}
\def\nln{{n_\mathrm{ln}}}
\def\nq{{n_\mathrm{eq}}}
\def\nelem{{n_\mathrm{e}}}
\def\nnode{{n_\mathrm{n}}}
\def\ndf{{n_\mathrm{dof}}}
\def\idf{{i_\mathrm{dof}}}
\def\ned{{n_\mathrm{ed}}}
\def\nee{{n_\mathrm{ee}}}
\def\iee{{i_\mathrm{ee}}}
\def\nint{{n_\mathrm{int}}}
\def\nquad{{n_\mathrm{quad}}}
\def\iquad{{i_\mathrm{quad}}}
\def\neemax{n_\mathrm{ee\_max}}
\def\npes{{n_\mathrm{PEs}}}
\def\Ot{\Omega_t}
\def\Gt{\Gamma_t}
\def\On{\Omega_n}
\def\Gn{\Gamma_n}
\newcommand{\Reynolds}{\mathrm{Re}}
\newcommand{\Prandtl}{\mathrm{Pr}}
\newcommand{\Froude}{\mathrm{Fr}}
\newcommand{\Strouhal}{\mathrm{St}}
\def\sbn{{\pmb n}}
\def\sbt{{\pmb t}}
\def\sbb{{\pmb b}}
\newcommand{\ba}{\mathbf{a}}
\newcommand{\bb}{\mathbf{b}}
\newcommand{\bc}{\mathbf{c}}
\newcommand{\bd}{\mathbf{d}}
\newcommand{\be}{\mathbf{e}}
\newcommand{\force}{\mathbf{f}}
\newcommand{\bg}{{\boldsymbol{\mathit{g}}}}
\newcommand{\bh}{{\boldsymbol{\mathit{h}}}}
\newcommand{\bi}{\mathbf{i}}
\newcommand{\bj}{\mathbf{j}}
\newcommand{\bk}{\mathbf{k}}
\newcommand{\bm}{\mathbf{m}}
\newcommand{\bn}{\mathbf{n}}
\newcommand{\bp}{\mathbf{p}}
\newcommand{\bq}{\mathbf{q}}
\newcommand{\br}{\mathbf{r}}
\newcommand{\bs}{\mathbf{s}}
\newcommand{\bt}{\mathbf{t}}
\newcommand{\bu}{\mathbf{u}}
\newcommand{\bv}{\mathbf{v}}
\newcommand{\bw}{\mathbf{w}}
\newcommand{\bx}{\mathbf{x}}
\newcommand{\by}{\mathbf{y}}
\newcommand{\bz}{\mathbf{z}}
\newcommand{\bA}{\mathbf{A}}
\newcommand{\bB}{\mathbf{B}}
\newcommand{\bC}{\mathbf{C}}
\newcommand{\bD}{\mathbf{D}}
\newcommand{\bE}{\mathbf{E}}
\newcommand{\bF}{\mathbf{F}}
\newcommand{\bG}{\mathbf{G}}
\newcommand{\bH}{\mathbf{H}}
\newcommand{\bI}{\mathbf{I}}
\newcommand{\bJ}{\mathbf{J}}
\newcommand{\bK}{\mathbf{K}}
\newcommand{\bL}{\mathbf{L}}
\newcommand{\bM}{\mathbf{M}}
\newcommand{\bN}{\mathbf{N}}
\newcommand{\bO}{\mathbf{O}}
\newcommand{\bP}{\mathbf{P}}
\newcommand{\bQ}{\mathbf{Q}}
\newcommand{\bR}{\mathbf{R}}
\newcommand{\bS}{\mathbf{S}}
\newcommand{\bT}{\mathbf{T}}
\newcommand{\bU}{\mathbf{U}}
\newcommand{\bV}{\mathbf{V}}
\newcommand{\bW}{\mathbf{W}}
\newcommand{\bX}{\mathbf{X}}
\newcommand{\bY}{\mathbf{Y}}
\newcommand{\bZ}{\mathbf{Z}}
\newcommand{\zero}{\mathbf{0}}
\newcommand{\bxi}{{\boldsymbol{\xi}}}
\newcommand{\balpha}{{\boldsymbol{\alpha}}}
\newcommand{\bomega}{{\boldsymbol{\omega}}}
\newcommand{\bchi}{{\boldsymbol{\chi}}}
\newcommand{\btheta}{{\boldsymbol{\theta}}}
%\newcommand{\bChi}{{\boldsymbol{\Chi}}}
\newcommand{\bepsilon}{{\boldsymbol{\epsilon}}}
\newcommand{\bPsi}{{\boldsymbol{\Psi}}}
\newcommand{\bphi}{{\boldsymbol{\phi}}}
\newcommand{\bpi}{{\boldsymbol{\pi}}}
\newcommand{\design}{\balpha}
\def\but{{\bf u}_{,t}}
\def\bvt{{\bf v}_{,t}}
\def\ST{{{\cal S}^h_\bT}}
\def\VT{{{\cal V}^h_\bT}}
\def\SH{{{\cal S}^h_H}}
\def\VH{{{\cal V}^h_H}}
\def\Su{{{\cal S}^h_\bu}}
\def\Vu{{{\cal V}^h_\bu}}
\def\SU{{{\cal S}^h_\bU}}
\def\VU{{{\cal V}^h_\bU}}
\def\Sv{{{\cal S}^h_\bv}}
\def\Vv{{{\cal V}^h_\bv}}
\def\Sp{{{\cal S}^h_p}}
\def\Vp{{{\cal V}^h_p}}
\def\Sh{{{\cal S}^h_\phi}}
\def\Vh{{{\cal V}^h_\phi}}
\def\Sun{{(\Su)_n}}
\def\Vun{{(\Vu)_n}}
\def\SHn{{(\SH)_n}}
\def\VHn{{(\VH)_n}}
\def\SUn{{(\SU)_n}}
\def\VUn{{(\VU)_n}}
\def\Svn{{(\Sv)_n}}
\def\Vvn{{(\Vv)_n}}
\def\Spn{{(\Sp)_n}}
\def\Vpn{{(\Vp)_n}}
\def\Shn{{(\Sh)_n}}
\def\Vhn{{(\Vh)_n}}
\newcommand{\strain}{{\boldsymbol{\varepsilon}}}
\newcommand{\stress}{{\boldsymbol{\sigma}}}
\newcommand{\blambda}{{\boldsymbol{\lambda}}}
\renewcommand{\div}{{\boldsymbol{\nabla}}}
\newcommand{\sbg}{\bg}
\newcommand{\sbh}{\bh}
\newcommand{\btau}{\boldsymbol{\tau}}
\newcommand{\tauT}{{\tau_\mathrm{\scriptscriptstyle CONS}}}
\newcommand{\tauu}{{\tau_\mathrm{\scriptscriptstyle MOM}}}
\newcommand{\taup}{{\tau_\mathrm{\scriptscriptstyle CONT}}}
\newcommand{\taudc}{{\tau_\mathrm{\scriptscriptstyle DC}}}
\newcommand{\tauadv}{{\tau_\mathrm{\scriptscriptstyle ADV}}}
\newcommand{\tausupg}{{\tau_\mathrm{\scriptscriptstyle SUPG}}}
\newcommand{\taupspg}{{\tau_\mathrm{\scriptscriptstyle PSPG}}}
\newcommand{\taugls}{{\tau_\mathrm{\scriptscriptstyle GLS}}}
\newcommand{\tauglsa}{{\tau_\mathrm{\scriptscriptstyle GLS1}}}
\newcommand{\tauglsb}{{\tau_\mathrm{\scriptscriptstyle GLS2}}}
\newcommand{\oneovermu}{\frac{1}{2 \mu_{1}}}
\newcommand{\lamovermu}{\frac{\lambda}{2 \mu_{1}}}
\newcommand{\lsqovermu}{\frac{\lambda^{2}}{2 \mu_{1}}}
\newcommand{\hash}{{\includegraphics[width=0.1in]{x.ps}}}
\newcommand{\hashblue}{{\includegraphics[width=0.1in]{x-blue.ps}}}
\newcommand{\subhash}{{\includegraphics[width=0.06in]{x.ps}}}
\newcommand{\subhashblue}{{\includegraphics[width=0.06in]{x-blue.ps}}}
\newcommand{\upup}           {{\includegraphics[width=0.1in]{h.ps}}}
\newcommand{\upupblue}       {{\includegraphics[width=0.1in]{h-blue.ps}}}
\newcommand{\upupred}        {{\includegraphics[width=0.1in]{h-red.ps}}}
\newcommand{\upupmagenta}    {{\includegraphics[width=0.1in]{h-magenta.ps}}}
\newcommand{\subupup}        {{\includegraphics[width=0.1in]{h.ps}}}
\newcommand{\subupupblue}    {{\includegraphics[width=0.06in]{h-blue.ps}}}
\newcommand{\subupupred}     {{\includegraphics[width=0.06in]{h-red.ps}}}
\newcommand{\subupupmagenta} {{\includegraphics[width=0.06in]{h-magenta.ps}}}

\def\half{\frac{1}{2}}
\def\us{u^*}	\def\bus{{\bu^*}}	\def\buh{{\bu^h}}
\def\udiv{\bu\!\cdot \div}	\def\usdiv{\bus\!\cdot \div}
\def\uhdiv{\buh\!\cdot \div}
\def\divu{\div\!\cdot\!\bu}	\def\divus{\div\!\cdot\!\bus}
\def\divuh{\div\!\cdot\!\buh} \def\divv{\div\!\cdot\!\bv}
\def\vatwo{\left[\begin{array}{c}v_1^a\\v_2^a\end{array}\right]}
\def\ubtwo{\left[\begin{array}{c}u_1^b\\u_2^b\end{array}\right]}
\def\va3{\left[\begin{array}{c}v_1^a\\v_2^a\\v_3^a\end{array}\right]}
\def\ub3{\left[\begin{array}{c}u_1^b\\u_2^b\\u_3^b\end{array}\right]}
\def\fb3{\left[\begin{array}{c}f_1^b\\f_2^b\\f_3^b\end{array}\right]}
\def\rs#1{{r_{#1}^*}}	\def\brs#1{{{\bf r}_{#1}^*}}
\def\gradv3{\left[\begin{array}{ccc}
	v_{1,x_1}&v_{1,x_2}&v_{1,x_3}\\
	v_{2,x_1}&v_{2,x_2}&v_{2,x_3}\\
	v_{3,x_1}&v_{3,x_2}&v_{3,x_3}\\
\end{array}\right]}
\def\strainu3{\left[\begin{array}{ccc}
	2 u_{1,x_1}&u_{1,x_2}+u_{2,x_1}&u_{1,x_3}+u_{3,x_1}\\
	u_{2,x_1}+u_{1,x_2}&2 u_{2,x_2}&u_{2,x_3}+u_{3,x_2}\\
	u_{3,x_1}+u_{1,x_3}&u_{3,x_2}+u_{2,x_3}&2 u_{3,x_3}
\end{array}\right]}
\def\divstrainv3{\left[\begin{array}{c}
	2v_{1,x_1x_1}+v_{1,x_2x_2}+ v_{1,x_3x_3}+v_{2,x_1x_2}+ v_{3,x_1x_3}\\
	 v_{1,x_2x_1}+v_{2,x_1x_1}+2v_{2,x_2x_2}+v_{2,x_3x_3}+ v_{3,x_2x_3}\\
	 v_{1,x_3x_1}+v_{2,x_3x_2}+ v_{3,x_1x_1}+v_{3,x_2x_2}+2v_{3,x_3x_3}
\end{array}\right]}
\def\divstrainu3{\left[\begin{array}{c}
	2u_{1,x_1x_1}+u_{1,x_2x_2}+ u_{1,x_3x_3}+u_{2,x_1x_2}+ u_{3,x_1x_3}\\
	 u_{1,x_2x_1}+u_{2,x_1x_1}+2u_{2,x_2x_2}+u_{2,x_3x_3}+ u_{3,x_2x_3}\\
	 u_{1,x_3x_1}+u_{2,x_3x_2}+ u_{3,x_1x_1}+u_{3,x_2x_2}+2u_{3,x_3x_3}
\end{array}\right]}
\def\gradq3{\left[\begin{array}{c}q_{x_1}\\q_{x_2}\\q_{x_3}\end{array}\right]}
\def\gradp3{\left[\begin{array}{c}p_{x_1}\\p_{x_2}\\p_{x_3}\end{array}\right]}
%
% stress components
%
\def\Ts{{T^*}}	\def\bTs{{\bT^*}}	\def\bTh{{\bT^h}}	\def\bSh{{\bS^h}}
\def\S#1{{\tilde{S}_{#1}}}
\def\T#1{{\tilde{T}_{#1}}}	\def\Ts#1{{\tilde{T}_{#1}^*}}
%
\def\Sa{\left[\begin{array}{c}\S{1}^a\\\S{2}^a\\\S{3}^a\end{array}\right]}
\def\Tb{\left[\begin{array}{c}\T{1}^b\\\T{2}^b\\\T{3}^b\end{array}\right]}
%
% convective derivatives of stress
%
\def\dusT{\left( \div \bus \cdot \bT + \bT \cdot (\div \bus)^T\right)}
\def\duTs{\left( \div \bu \cdot \bTs + \bTs \cdot (\div \bu)^T\right)}
\def\duTh{\left( \div \buh \cdot \bTh + \bTh \cdot (\div \buh)^T\right)}
\def\dusTs{\left( \div \bus \cdot \bTs + \bTs \cdot (\div \bus)^T\right)}
\def\duS{\left( \div \bu \cdot \bS + \bS \cdot (\div \bu)^T\right)}
\def\duSh{\left( \div \buh \cdot \bSh + \bSh \cdot (\div \buh)^T\right)}
\def\dusS{\left( \div \bus \cdot \bS + \bS \cdot (\div \bus)^T\right)}
\def\triangledown{\mathchar"0235 }
\def\bDT{\overtriangle{\bT}}
\def\bDS{\overtriangle{\bS}}
%
% common VF expressions
%

\newcommand{\piola}{\mathcal{J}{\bf F^{-1}\tilde{T}F^{-T} }}
\newcommand{\rhou}{\left( \rho \bu \right)}
\newcommand{\rhoe}{\left( \rho e \right)}
\newcommand{\rhouu}{\left( \rho \bu \bu \right)}
\newcommand{\rhoeu}{\left( \rho e \bu \right)}
\newcommand{\drdt}{\frac{\partial \rho}{\partial t}}
\newcommand{\drudt}{\frac{\partial \rhou}{\partial t}}
\newcommand{\dredt}{\frac{\partial \rhoe}{\partial t}}
\newcommand{\dudt}{\frac{\partial \bu}{\partial t}}
\newcommand{\dudtsq}{\frac{\partial^2 \bu}{\partial t^2}}
\newcommand{\dUdt}{\frac{\partial \bU}{\partial t}}
\newcommand{\gradu}{\div \bu}
\newcommand{\gradut}{\left( \div \bu \right)^{T}}
\newcommand{\umag}{\|\bu\|}
\newcommand{\dudx}{\frac{\partial u_{1}}{\partial x_{1}}}
\newcommand{\dudy}{\frac{\partial u_{1}}{\partial x_{2}}}
\newcommand{\dudz}{\frac{\partial u_{1}}{\partial x_{3}}}
\newcommand{\dvdx}{\frac{\partial u_{2}}{\partial x_{1}}}
\newcommand{\dvdy}{\frac{\partial u_{2}}{\partial x_{2}}}
\newcommand{\dvdz}{\frac{\partial u_{2}}{\partial x_{3}}}
\newcommand{\dwdx}{\frac{\partial u_{3}}{\partial x_{1}}}
\newcommand{\dwdy}{\frac{\partial u_{3}}{\partial x_{2}}}
\newcommand{\dwdz}{\frac{\partial u_{3}}{\partial x_{3}}}


\begin{document}

\title{Myocardial Motion Estimation} \author{} \maketitle

This article attempts to lay the mathematical foundation for myocardial motion
estimation posed as a pde constrained non-linear optimization problem. We
attempt to maximize the similarity between the estimated images and the
observed MR cine images, constrained by a mechanical model of the heart. We
initially assume a homgeneous linear elastic model of the heart, and derive
the optimality conditions for it. Damping is also ignored in the begining.
Damping and non-linear elasticity in the form of a hyper-elastic neo-hookean
material shall be incorporated later if required. We ignore the boundary
conditions initially, they shall be incorporated later into the formulation.

% Need to change the force formulation in these problems to one where $f$ is a scalar and the orientations are obtained from the DTI fiber orientations.

\section{Linear Elastodynamics}

We assume that the MR image occupies the region $\Omega \in \mathbb{R}^3$ in its reference state (end-diastole) at time $t=0$. The boundary $\Gamma=\partial\Omega$ has the outward unit normal given by $\bn$, the displacement vector is represented by $\bu$, and the velocity by $\bv$. The myocardium is assumed to be made of an elastic material and is subject to body force $\force$ per unit volume.

The strong form of the equations of linear elastodynamics are written as:
\begin{eqnarray}
\rho\dudtsq &=& \div\cdot\stress + \force \qquad \mbox{in} ~\Omega\times]0,T[, \\
\stress \bn &=& 0, \qquad\qquad~~~~ \mbox{in} ~\Gamma\times]0,T[, \nonumber\\
\bu(t=0) &=& \bu_0, \qquad\qquad~~ \mbox{in} ~\Omega \nonumber\\
\dot{\bu}(t=0) &=& \bv_0, \qquad\qquad~~ \mbox{in} ~\Omega \nonumber
\end{eqnarray}
where $\bu_0$ is the initial displacement, $\bv_0$ is the initial velocity, $\stress$ is the stress tensor and $\div\cdot\stress$ denotes the divergence of $\stress$. Assuming that the material is isotropic and homogeneous, one may express the stress tensor as
\begin{equation}
\label{eq:stressstrain}
\stress = \lambda~\mbox{tr}~\strain\bI + 2\mu\strain,
\end{equation}
in terms of the Lam\'{e} constants $\lambda$ and $\mu$, the identity tensor $\bI$, and the infinitesimal strain tensor $\strain$. The strain is defined as
\begin{equation}
\label{eq:strain}
\strain := \frac{1}{2}\left[ \gradu + \gradut \right] := \div_s\bu~,
\end{equation} 
where $\gradu$ is the gradient operator expessed in Cartesian component form as
\[
[\gradu] = \left[ \begin{array}{c}u_1\\u_2\\u_3\end{array}\right] 
\left[\begin{array}{ccc} \dfrac{\partial}{\partial x} & \dfrac{\partial}{\partial y} & \dfrac{\partial}{\partial z} \end{array}\right] =
\left[
\begin{array}{ccc} 
u_{1,1} & u_{1,2} & u_{1,3} \\
u_{2,1} & u_{2,2} & u_{2,3} \\
u_{3,1} & u_{3,2} & u_{3,3}
\end{array}
\right] ~.
\]
Using equation (\ref{eq:strain}), the components of the strain tensor are
\[
[\strain] = \left[
\begin{array}{ccc} 
u_{1,1} & \frac{1}{2}(u_{1,2}+u_{2,1}) & \frac{1}{2}(u_{1,3}+u_{3,1}) \\
\frac{1}{2}(u_{1,2}+u_{2,1}) & u_{2,2} & \frac{1}{2}(u_{2,3}+u_{3,2}) \\
\frac{1}{2}(u_{1,3}+u_{3,1}) & \frac{1}{2}(u_{2,3}+u_{3,2}) & u_{3,3}
\end{array}
\right] ~.
\]

%% Variational formulation
Let $\bw$ denote the variations, and $\bw \in \mathcal{V}$ the variation space.
Then the weak form can be written as
\begin{equation}
\int_{\Omega} \bw\cdot(\rho\dudtsq-\div\cdot\stress -\force)~d\Omega + \int_{\Gamma} \bw\cdot\stress\bn~d\Gamma = 0.
\end{equation}
Using the Einsteinian summation convention, we can write it as,

\begin{eqnarray}
  0 & = & \int_{\Omega} w_i ( \rho u_{i, t t} - \sigma_{i j, j} - f_i )~d\Omega + \int_{\Gamma} w_i \sigma_{ij} n_j~d \Gamma \nonumber\\
	& = & \int_{\Omega} w_i \rho_i u_{i, t t}~d\Omega - 	
	\left(	\int_{\Omega} w_i\sigma_{ij,j}~d\Omega \right) -
  \int_{\Omega} w_i f_i~d\Omega + \int_{\Gamma} w_i \sigma_{ij} n_j~d\Gamma \nonumber\\
  & = & \int_{\Omega} w_i \rho_i u_{i, t t}~d\Omega - 	
	\left(	\int_{\Omega} (w_i\sigma_{ij})_{,j}~d\Omega - \int_{\Omega} w_{i,j}\sigma_{ij}~d\Omega \right) -
  \int_{\Omega} w_i f_i~d\Omega + \int_{\Gamma} w_i \sigma_{ij} n_j~d\Gamma \nonumber\\
	& = & \int_{\Omega} w_i \rho_i u_{i, t t}~d\Omega - 	
	\left( \int_{\Gamma} w_i \sigma_{ij} n_j~d\Gamma - \int_{\Omega} w_{i, j}\sigma_{ij}~d\Omega \right) -
  \int_{\Omega} w_i f_i~d\Omega + \int_{\Gamma} w_i \sigma_{ij} n_j~d\Gamma \nonumber\\
  & = & \int_{\Omega} w_i \rho_i u_{i, t t}~d\Omega + \int_{\Omega} w_{( i, j )} \sigma_{ij}~d\Omega - \int_{\Omega} w_i f_i~d\Omega,
\end{eqnarray}
where use is made of integration by parts and the divergence theorem. It follows that,
\begin{equation}
\label{eq:weakForm}
\int_{\Omega} \bw\cdot\rho\ddot{\bu} ~d\Omega - \int_{\Omega}\div_s\bw:\stress ~d\Omega = \int_{\Omega} \bw\cdot\force ~d\Omega ~,
\end{equation}
where $\div_s\bw:\stress$ denotes the contraction of the tensors $\div_s\bw$ and $\stress$, expressed in component form as $\div_s\bw:\stress=w_{i,j}\sigma_{ij}$.

Equation (\ref{eq:weakForm}) motivates us to define the following bilinear forms:
\begin{eqnarray}
  a (\bw,\bu) & = & -\int_{\Omega}\div_s\bw:\stress ~d\Omega, \label{eq:bilinear} \\ %= \int_{\Omega} w_{( i, j )} \sigma_{ij}~d\Omega\\
  (\bw,\force) & = & \int_{\Omega} \bw\cdot\force ~d\Omega, ~\mbox{and}~ \\ %= \int_{\Omega} w_i f_i~d\Omega \nonumber\\
  (\bw, \rho \ddot{\bu} ) & = & \int_{\Omega} \bw\cdot\rho\ddot{\bu} ~d\Omega. % = \int_{\Omega} w_i \rho u_{i, t t} d \Omega \nonumber
\end{eqnarray}
The corresponding weak formulation can be written as:

Given $\force, \bu_0 \tmop{and} \bv_0$, find $\bu( t ) \in \mathcal{S}_t, t \in [ 0, T ]$,
such that for all $\bw \in \mathcal{V}$, such that 
\begin{eqnarray}
  (\bw, \rho \ddot{\bu} ) + a (\bw,\bu) &
  = & (\bw,\force), \\
  (\bw, \rho \bu( 0 ) ) & = & (\bw, \rho \bu_0 ), \nonumber\\
  (\bw, \rho \dot{\bu} ( 0 ) ) & = & (\bw, \rho \tmmathbf{v}_0 ) \nonumber.
\end{eqnarray}

In order to simply the expressions, we express the components of tensorial quantities such as $\div_s\bw$ and $\stress$ in vector form. In particular we define the strain vector as,  

\[ 
	\strain(\bu) = 
	\left\{ \begin{array}{c}
	\epsilon_{11} \\ \epsilon_{22} \\ \epsilon_{33} \\
	2\epsilon_{12} \\ 2\epsilon_{23} \\ 2\epsilon_{31}
	\end{array} \right\}  =
	\left\{ \begin{array}{c}
     u_{1, 1}\\
     u_{2, 2}\\
     u_{3, 3}\\
     u_{2, 3} + u_{3, 2}\\
     u_{1, 3} + u_{3, 1}\\
     u_{1, 2} + u_{2, 1}
   \end{array} \right\} 
\]

Likewise, the stress tensor can be written in vector form as,

\[
\stress = \left\{ \begin{array}{c}
	\sigma_{11} \\ \sigma_{22} \\ \sigma_{33} \\
	\sigma_{12} \\ \sigma_{23} \\ \sigma_{31} \end{array} \right\} 
\]

The stress-strain law (\ref{eq:stressstrain}) can be written using the vector convention as
\begin{equation}
\label{ref:stressstrainMat}
[\stress] = [\bD][\strain],
\end{equation} 

where $[\bD]$ is a ($6\times6$) elasticity matrix such that
\begin{equation}
  \bD= \left[ \begin{array}{cccccc}
    \lambda + 2 \mu & \lambda & \lambda & 0 & 0 & 0\\
    \lambda & \lambda + 2 \mu & \lambda & 0 & 0 & 0\\
    \lambda & \lambda & \lambda + 2 \mu & 0 & 0 & 0\\
    0 & 0 & 0 & \mu & 0 & 0\\
    0 & 0 & 0 & 0 & \mu & 0\\
    0 & 0 & 0 & 0 & 0 & \mu
  \end{array} \right]
\end{equation}

Since the matrix $[\bD]$ is always symmetric, it follows that the integrand of the bilinear form in (\ref{eq:bilinear}) can be written with the aid of (\ref{ref:stressstrainMat}) as

\[
\div_s\bw:\stress = [\strain(\bw)][\bD][\strain(\bu)] := \strain(\bw)\cdot\bD\strain(\bu) ,
\]
which shows that the bilinear form in (\ref{eq:bilinear}) in indeed symmetric. 

\subsection{Semidiscrete Galerkin formulation of elastodynamics}

Given $\force, \bu_0, \tmop{and} \dot{\bu_0}$, find $\bu^h =\bv^h
+\tmmathbf{g}^h,\bu^h ( t ) \in \mathcal{S}_t^h$, such that for all
$\bw^h \in \mathcal{V}^h$,
\begin{eqnarray}
  (\bw^h, \rho \ddot{\tmmathbf{v}}^h ) + a
  (\bw^h,\tmmathbf{v}^h ) & = & (\bw^h, f ) 
- (\bw^h, \rho \ddot{\tmmathbf{g}}^h ) - a (\bw^h,\tmmathbf{g}^h ) \nonumber\\
  (\bw^h, \rho \tmmathbf{v}^h ( 0 ) ) & = & (\bw^h, \rho
  \bu_0 ) - (\bw^h, \rho \tmmathbf{g}^h ( 0 ) ) \nonumber\\
  (\bw^h, \rho \dot{\tmmathbf{v}}^h ( 0 ) ) & = & (\bw^h,
  \rho \dot{\bu}_0 ) - (\bw^h, \rho \dot{\tmmathbf{g}}^h ( 0
  ) ) 
\end{eqnarray}
The representations of $\bw^h,\tmmathbf{v}^h \tmop{and}
\tmmathbf{g}^h$ are given by
\begin{eqnarray}
  \bw^h (\tmmathbf{x}, t ) = w_i^h (\tmmathbf{x}, t )\tmmathbf{e}_i &
  = & \sum_{A \in \eta - \eta_{q_i}} N_A (\tmmathbf{x}) c_{i A} ( t
  )\tmmathbf{e}_i \nonumber\\
  \tmmathbf{v}^h (\tmmathbf{x}, t ) = v_i^h (\tmmathbf{x}, t )\tmmathbf{e}_i &
  = & \sum_{A \in \eta - \eta_{g_i}} N_A (\tmmathbf{x}) d_{i A} ( t
  )\tmmathbf{e}_i \nonumber\\
  \tmmathbf{g}^h (\tmmathbf{x}, t ) = g_i^h (\tmmathbf{x}, t )\tmmathbf{e}_i &
  = & \sum_{A \in \eta_{g_i}} N_A (\tmmathbf{x}) g_{i A} ( t )\tmmathbf{e}_i 
\end{eqnarray}
Substituting (14) into (13) we get, (ignoring $\tmmathbf{e}_i, \tmop{and}
\tmmathbf{e}_j$ for clarity,

\begin{eqnarray*}
& & \left( \sum_{A \in \eta_{\tmop{int}}} N_A (\tmmathbf{x}) c_{i A} ( t ),
   \sum_{B \in \eta_{\tmop{int}}} N_B (\tmmathbf{x}) \rho \ddot{d_{}}_{j B} (
   t ) \right) + a \left( \sum_{A \in \eta_{\tmop{int}}} N_A (\tmmathbf{x})
   c_{i A} ( t ), \sum_{B \in \eta_{\tmop{int}}} N_B (\tmmathbf{x}) d_{i B} (
   t ) \right) \\
&=& \left( \sum_{A \in \eta_{\tmop{int}}} N_A (\tmmathbf{x}) c_{i
   A} ( t ),\force \right) + \left( \sum_{A \in \eta_{\tmop{int}}} N_A
   (\tmmathbf{x}) c_{i A} ( t ),\mathfrak{h} \right)_{\Gamma} -   
   \left( \sum_{A \in \eta_{\tmop{int}}} N_A (\tmmathbf{x}) c_{i A} ( t ),
   \sum_{B \in \eta_{q_i}} N_B \rho \ddot{g}_{i B} ( t ) \right) \\
&-& a\left(  \sum_{A \in \eta_{\tmop{int}}} N_A (\tmmathbf{x}) c_{i A} ( t ), \sum_{B
   \in \eta_{q_i}} N_B g_{i B} ( t ) \right) 
\end{eqnarray*}

This gives us a set of $3 \eta_{\tmop{int}} = 3 ( \eta - \eta_{q_i} )$
equations, where $\eta$ is the total number of nodes,
\begin{eqnarray*}
& & \sum^{n_{\tmop{dof}}}_{j = 1} \sum_{B \in \eta_{\tmop{int}}} ( N_A
  \tmmathbf{e}_i, N_B \tmmathbf{e}_j ) \rho \ddot{d}_{j B} ( t ) + \sum_{j =
  1}^{n_{\tmop{dof}}} \sum_{B \in \eta_{\tmop{int}}} a ( N_A \tmmathbf{e}_i,
  N_B \tmmathbf{e}_j ) d_{j B} ( t ) \\
&=& ( N_A \tmmathbf{e}_i,\force) + ( N_A \tmmathbf{e}_j,\mathfrak{h})_{\Gamma}  - \sum^{n_{\tmop{dof}}}_{j = 1}
  \sum_{B \in \eta_{q_j}} ( N_A \tmmathbf{e}_i, N_B \tmmathbf{e}_j ) \rho
  \ddot{g}_{j B} ( t ) - \sum_{B \in \eta_{q_j}} a ( N_A \tmmathbf{e}_i, N_B
  \tmmathbf{e}_j ) g_{j B} ( t )
\end{eqnarray*}


For the case of homogeneous dirichlet boundary conditions, $\tmmathbf{g}= 0,
\tmop{and} \eta_{q_i} = 0$, therefore (10) reduces to a set of $3 \eta$
equations in $3 \eta$ unknowns,
\begin{equation}
  \sum^{n_{\tmop{dof}}}_{j = 1} \sum_{B \in \eta} ( N_A \tmmathbf{e}_i, N_B
  \tmmathbf{e}_j ) \rho \ddot{d}_{j B} ( t ) + \sum^{n_{\tmop{dof}}}_{j = 1}
  \sum_{B \in \eta} a ( N_A \tmmathbf{e}_i, N_B \tmmathbf{e}_j ) d_{j B} ( t )
  = ( N_A \tmmathbf{e}_i,\force) + ( N_A
  \tmmathbf{e}_i,\mathfrak{h})_{\Gamma}
\end{equation}
In order to derive the mass matrix, the global stiffness matrix and the force
vector, we need to specify the global ordering of equations. This shall be
explained in detail later, for now we assume we have a function
{\tmstrong{id(i,A)}} that takes the degree of freedom and the global node
number as input and returns the global equation number. Using this we can
write the {\tmstrong{matrix problem}} as:

%\begin{tabular}{|c|}
%  \hline \\
%  \begin{eqnarray}
%    \tmmathbf{M} \ddot{\tmmathbf{d}} +\tmmathbf{K} \tmmathbf{d} & = &
%    \force \nonumber\\
%    \tmmathbf{d}( 0 ) & = & \tmmathbf{d}_0 \nonumber\\
%    \dot{\tmmathbf{d}} ( 0 ) & = & \dot{\tmmathbf{d}}_0 
%  \end{eqnarray}
%  \hline
%\end{tabular}

where the mass matrix,
\[ \tmmathbf{M}= \bigwedge_{e = 1}^{n_{\tmop{el}}} (\tmmathbf{m}^e ) \]
here, $\bigwedge$is the matrix assembly operator, and the elemental mass
matrix, $\tmmathbf{m}^e$ is given in terms of nodal submatrices as,
\begin{eqnarray}
  \tmmathbf{m}^e & = & \left[ m_{p q}^e \right] \nonumber\\
  m_{p q}^e & = & \delta_{i j} \int_{\Omega_e} N_a \rho N_b d \Omega 
\end{eqnarray}
\begin{notation}
  In all these definitions, $n_{\tmop{en}}$ is the number of element nodes,
  which for the trilinear hexahedral element is 8; $n_{\tmop{ed}}$ is the
  number of element degrees of freedom per node, which is 3 in our case. Also
  $n_{\tmop{ee}}$ stands for the number of element equations, which is
  $n_{\tmop{ed}} n_{\tmop{en}}$.
\end{notation}

\begin{notation}
  For the elemental mass matrix, $1 \leqslant p, q \leqslant n_{\tmop{ee}} =
  n_{\tmop{ed}} n_{\tmop{en}}$. Therefore in our case the elemental mass
  matrix shall be $24 \times 24$. For the nodal submatrices, indices  $p =
  n_{\tmop{ed}} ( a - 1 ) + i$, and $q = n_{\tmop{ed}} ( b - 1 ) + j$. 
\end{notation}

Similarly, the stiffness matrix can be written as,
\begin{eqnarray}
  \tmmathbf{K} & = & \bigwedge_{e = 1}^{n_{\tmop{el}}} (\tmmathbf{k}^e )
  \nonumber\\
  \tmmathbf{k}^e & = & \left[ k_{p q}^e \right] \nonumber\\
  k_{p q}^e & = & \tmmathbf{e}_i^T \int_{\Omega_e} \tmmathbf{B}_a^T
  \tmmathbf{D} \tmmathbf{B}_a d \Omega \tmmathbf{e}_j 
\end{eqnarray}
where the matrices $\tmmathbf{D}$ was defined earlier (7) and $\tmmathbf{B}_a$
is given by,
\begin{eqnarray*}
  \tmmathbf{B}_a & = & \left[ \begin{array}{ccc}
    N_{a, x} & 0 & 0\\
    0 & N_{a, y} & 0\\
    0 & 0 & N_{a, z}\\
    0 & N_{a, z} & N_{a, y}\\
    N_{a, z} & 0 & N_{a, x}\\
    N_{a, y} & N_{a, x} & 0
  \end{array} \right]
\end{eqnarray*}
We can now proceed to write the right hand side of our equation, i.e., the
force vector. This can be written as,
\begin{eqnarray}
  \force( t ) & = & \force_{\tmop{nodal}} ( t ) +
  \bigwedge_{e = 1}^{n_{\tmop{el}}} (\force^e ( t ) ) \nonumber\\
  \force^e & = & \{ f_p^e \} \nonumber\\
  f_p^e & = & \int_{\Omega_e} N_a f_i d \Omega +
  \int_{\Gamma_{\mathfrak{h}_i}} N_a \mathfrak{h}_i d \Gamma 
\end{eqnarray}
Also, we get the expressions for the initial conditions as,
\begin{eqnarray*}
  \tmmathbf{d}_0 & = & \tmmathbf{M}^{- 1} \bigwedge_{e =
  1}^{n_{\tmop{el}}} ( \hat{\tmmathbf{d}}^e )\\
  \hat{\tmmathbf{d}}^e & = & \{ \hat{d}_p^e \}\\
  \hat{d}_p^e & = & \int_{\Omega_e} N_a \rho u_{0 i} d \Omega\\
  \dot{\tmmathbf{d}_0} & = & \tmmathbf{M}^{- 1} \bigwedge_{e =
  1}^{n_{\tmop{el}}} ( \widehat{\dot{\tmmathbf{d}^e}}^{} )\\
  \widehat{\dot{\tmmathbf{d}^e}} & = & \{ \widehat{\dot{d_{}^{}}} _p^e \}\\
  \widehat{\dot{d_{}^{}}} _p^e & = & \int_{\Omega_e} N_a \rho \dot{u}_{0 i} d
  \Omega
\end{eqnarray*}
Of course under the assumption that the heart is stationary at the end of
diastole, which is our reference frame, $u_{0 i} = \dot{u_{0 i}} = 0$ and
therefore $\tmmathbf{d}_0 = \dot{\tmmathbf{d}_{}}_0 =\tmmathbf{0}$.

\subsection{Solving the forward problem: Newmark Scheme}

In order to solve the forward problem, we use the Newmark method. We use the
average acceleration or trapezoidal rule, which uses the values $\beta = 1 / 4
\tmop{and} \gamma = 1 / 2$. We first define the predictors,
\begin{eqnarray}
  \tilde{\tmmathbf{d}}_{n + 1} & = & \tmmathbf{d}_n + \Delta t\tmmathbf{v}_n +
  \frac{\Delta t^2}{4} \tmmathbf{a}_n \nonumber\\
  \tilde{\tmmathbf{v}}_{n + 1} & = & \tmmathbf{v}_n + \frac{\Delta t}{2}
  \tmmathbf{a}_n 
\end{eqnarray}
We can determine the acceleration at the next time instant by solving,
\begin{equation}
  \left( \tmmathbf{M}+ \frac{\Delta t^2}{2} \tmmathbf{K} \right)
  \tmmathbf{a}_{n + 1} =\force_{n + 1} -\tmmathbf{K}
  \tilde{\tmmathbf{d}}_{n + 1}
\end{equation}
We solve this to obtain $\tmmathbf{a}_{n + 1}$, and then obtain the values of
$\tmmathbf{d}_{n + 1} \tmop{and} \tmmathbf{v}_{n + 1}$ using the correctors
\begin{eqnarray}
  \tmmathbf{d}_{n + 1} & = & \tilde{\tmmathbf{d}}_{n + 1} + \frac{\Delta
  t^2}{4} \tmmathbf{a}_{n + 1} \nonumber\\
  \tmmathbf{v}_{n + 1} & = & \tilde{\tmmathbf{v}}_{n + 1} + \frac{\Delta t}{2}
  \tmmathbf{a}_{n + 1} 
\end{eqnarray}

\section{Optimization Problem}

The optimization problem can be formulated as follows,

minimize:
\[ \mathcal{J}(\bu, f ) = \frac{1}{2} \int^T_0
   \int_{\Omega} (\bu-\tmmathbf{z})^2 d \Omega d \tau +
   \frac{b}{2} \int^T_0 \int_{\Omega} \force^2 d \Omega d
   \tau \]
subject to:
\[ \begin{array}{lll}
     (\bw, \rho \ddot{\bu} ) + a (\bw,\bu)
     & = & (\bw,\force) + (\bw,\tmmathbf{h})_{\Gamma}
     \forall \bw \in \mathcal{V}
   \end{array} \]


We write the lagrange functional $\mathcal{L}(\bu,\force, \xi
)$ by introducing the lagrange multiplier $\xi$. The lagrange functional is
expressed as,
\begin{eqnarray*}
  \mathcal{L}(\bu, f, \xi ) & = & \int_0^T \left( \frac{1}{2}
  \int_{\Omega} (\bu-\tmmathbf{z})^2 d \Omega + \frac{b}{2}
  \int_{\Omega} \force^2 d \Omega \right) d \tau\\
  &  & - \int_0^T \left( (\tmmathbf{\xi}, \rho \ddot{\bu} ) +
  a (\tmmathbf{\xi},\bu) - (\tmmathbf{\xi},\force) -
  (\tmmathbf{\xi},\tmmathbf{h})_{\Gamma} \right) d \tau
\end{eqnarray*}
the optimization problem is one of finding $(\bu, f, \xi )$ such that
$\mathcal{L}(\bu, f, \xi )$ is stationary. This gives rise to three
conditions; the first is the constraint or the state equation, which in our
case is the weak form of the forward problem, i.e., find $\bu( t )
\in \mathcal{S}_t, t \in [ 0, T ]$, such that
\[ \begin{array}{lll}
     (\bw, \rho \ddot{\bu} ) + a (\bw,\bu)
     & = & (\bw,\force) + (\bw,\tmmathbf{h})_{\Gamma}
     \forall \bw \in \mathcal{V}
   \end{array} \]
Computing the differential with respect to $\bu$ gives us the adjoint
or the co-state equation. The adjoint equation can be written as, find
$\bu( t ) \in \mathcal{S}_t, t \in [ 0, T ]$, such that,
\[ ( (\bu-\tmmathbf{z}),\bw) = (\tmmathbf{\xi}, \rho
   \ddot{\bw} ) + a (\tmmathbf{\xi},\bw) \forall
   \bw \in \mathcal{V} \]
Similarly obtaining the derivatives with respect to the controls or the forces
$f$ in our case we obtain the optimality condition, which can be written as,
\[ ( b\force,\bw) = - (\tmmathbf{\xi},\bw) \forall
   \bw \in \mathcal{V} \]

In the descretized form the optimization problem can be written as,

\begin{equation}
\min_{\bu,\force,\bxi} \mathcal{L} (\bu, \force, \bxi) = \min_{\bu,\force,\bxi} \frac{1}{2} \Vert Q\bu - \bz \Vert^2 + \frac{\beta}{2} \force^TW\force + \bxi^T(L\bu - \force),
\end{equation}
where $Q$ corresponds to the interpolation matrix which uses the values of $\bu$ obtained by solving the forward problem to warp the end-diastole observation $\bz_0$ to obtain results at all points where the observations $\bz$ are available. $W$ is the smoothing operator in the objective function (identity matrix for the current case) and $L$ is the linear elasticity operator described earlier. KKT optimality conditions for the above problem are given by:

\begin{eqnarray*}
Q^T(Q\bu - \bz) + L^T\bxi&=& 0, \\
\beta W\force - \bxi &=& 0, \\
L\bu - \force &=& 0. 
\end{eqnarray*}

From the above equations we can derive the reduced gradient $g$ and the reduced Hessian $H$. The reduced gradient is given by,

\begin{eqnarray}
\bxi &=& -L^{-T}(Q^T(Q\bu - \bz)), \\
g &=& \beta W\force + L^{-T}(Q^T(Q\bu - \bz)) = 0. 
\end{eqnarray}

%In the above equation $-L^{-1}C$ is the sensitivity matrix ($S$) and as such it represents the variation of the field with respect to the force\footnote{the controls in general} $\force$. Using $S$ instead of $-L^{-1}C$ 

Using the expression for the reduced gradient, we can evaluate the reduced Hessian which is the derivative of the reduced gradient with respect to the controls ($\force$),
\begin{equation}
H = \beta W + L^{-T}Q^TQL^{-1}.
\end{equation}
 
%\subsection{Solving it ...}
%
%The optimization algorithm can be described as follows, starting with an
%initial guess for the forces, $f^{( 0 )}$,
%\begin{enumerateroman}
%  \item Solve the forward problem, $F (\bu^{( m )}, f^{( m )} ) = 0$
%  to obtain the displacements $\bu^{( m )}$.
%  
%  \item Compute the gradient of the cost functional, $\frac{D\mathcal{J}}{D f}
%  (\bu^{( m )}, f^{( m )} )$.
%  
%  \item Compute the step, $\delta f^{( m )}$ based on the gradient.
%  
%  \item $f^{( m + 1 )} \longleftarrow f^{( m )} + \delta f^{( m )}$ 
%\end{enumerateroman}
%We compute the gradient of the cost functional, by the adjoint method.

\end{document}
