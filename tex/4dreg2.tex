\documentclass{article}
\usepackage{amsmath,epsfig,calc,capt-of,ifthen}

%%%%%%%%%% Start TeXmacs macros
\newcommand{\tmfloatcontents}{}
\newlength{\tmfloatwidth}
\newcommand{\tmfloat}[5]{
  \renewcommand{\tmfloatcontents}{#4}
  \setlength{\tmfloatwidth}{\widthof{\tmfloatcontents}+1in}
  \ifthenelse{\equal{#2}{small}}
    {\ifthenelse{\lengthtest{\tmfloatwidth > \linewidth}}
      {\setlength{\tmfloatwidth}{\linewidth}}{}}
    {\setlength{\tmfloatwidth}{\linewidth}}  \begin{minipage}[#1]{\tmfloatwidth}
    \begin{center}
      \tmfloatcontents
      \captionof{#3}{#5}
    \end{center}
  \end{minipage}}
\newcommand{\tmop}[1]{\operatorname{#1}}
\newenvironment{itemizedot}
  {\begin{itemize}\renewcommand{\labelitemi}{$\bullet$}\renewcommand{\labelitemii}{$\bullet$}\renewcommand{\labelitemiii}{$\bullet$}\renewcommand{\labelitemiv}{$\bullet$}}{\end{itemize}}
\newenvironment{itemizearrow}
  {\begin{itemize}\renewcommand{\labelitemi}{$\rightarrow$}\renewcommand{\labelitemii}{$\rightarrow$}\renewcommand{\labelitemiii}{$\rightarrow$}\renewcommand{\labelitemiv}{$\rightarrow$}}{\end{itemize}}
\newcommand{\tmem}[1]{{\em #1\/}}
%%%%%%%%%% End TeXmacs macros

\begin{document}

\section{4D Registration}

We define two 4D sequences, the template $T$ and the subject $S$. The template
can be extensively preprocessed as it is a one time process. The goal of this
document is to formalize the 4D registration process starting from the two 4D
sequences. The 4D registration problem formulation is shown graphically in
figure 1.

\tmfloat{h}{big}{figure}{\epsfig{file=}}{Formulation of the 4D registration
problem}

\subsection{Spatio-Temporal Affine Registration}

We first perform an affine registration between the template and the subject.
We solve for a full 3D (spatial) affine transformation and for a temporal
translation and a scale. We do not permit rotation or shear between the
spatial and temporal coordinates as it does not make physical sense. The
process is done in three steps. Normalized mutual information is used as the
similarity measure. First we estimate the 3D affine transformation $A (
\vec{x} )$ by performing a 3D affine registration between the two end-diastole
images. Once this is estimated we then estimate the temporal shift and scale
of the subject with respect to the template, using the estimated spatial
transformation. The temporal normalization aligns the two sequences and makes
them both have the same number of frames. We follow this with a second set of
spatial registration. The difference in this case is that the similarity is
evaluated over all image pairs in the sequence. Since the temporal
normalization has been performed, the image pairs correspond to each other. We
iterate over the last two steps till they converge and we have the final 4D
affine transformation.

In the following steps it is assumed that the subject $S$ has been warped to
the template space using the results of the affine registration. Both the
sequences have $N$ frames each at the end of the affine registration.

\subsection{Attribute Vector}

The represent the attribute vector as $\vec{A} = ( \vec{W}, \vec{G} )$, where
$\vec{W}$ is the wavelet attribute at location $( \vec{x}, t )$ and $\vec{G}$
is the geometric attribute vector. The geometric attribute vector is made up
of the image intensity at different scales, graident, and the edge distance
transform.

\subsection{Combined motion field extraction and 4D registration}

We start with the images $T$ and $S$. Both are 4D image sequences and the
intensity at any location $( \vec{x}, t )$ can be accessed by, $T ( \vec{x}, t
)$ and $S ( \vec{x}, t )$. The attribute vector $\vec{A}$ at a given location
$( \vec{x}, t )$  is made of the two components, $\vec{W}$ and $\vec{G}$,
which are given by:
\begin{description}
  \item[Template] The wavelet attribute vector is represented by $\vec{W}_T (
  \vec{x}, t ) \tmop{and} \tmop{the} \tmop{geometric} \tmop{attribute}
  \tmop{vector} \tmop{by} \vec{G}_T ( \vec{x}, t )$.
  
  \item[Subject] The wavelet attribute vector is represented by $\vec{W}_S (
  \vec{x}, t )$ and the geometric attribute vector by $\vec{G}_S ( \vec{x}, t
  )$
\end{description}
We also define the following transformations:
\begin{itemizedot}
  \item $M_T ( \vec{x}, t )$: The motion field defined over the template
  space. The transformation maps a point $\vec{x}$ in the end-diastole frame
  of the template to its corresponding point in frame at time $t$.
  
  \item $M_S ( \vec{x}, t )$: The motion field defined over the subject space.
  The transformation maps a point $\vec{x}$ in the end-diastole frame of the
  subject to its corresponding point in frame at time $t$.
  
  \item $H_{T \leftarrow S} ( \vec{x}, t )$: The 4d transformation that maps a
  point $( \vec{x}, t )$ from the subject to the template space.
\end{itemizedot}
These transformations are illustrated in figure 1.

\subsubsection{Step 1: Estimating the Template Motion Field $M_T ( \vec{x}
\text{}$, t)}

This step needs to be done only once and can be done offline and the result
can be stored. A sparse set of focus points $\vec{p} \in \Omega_T ( \vec{x} )$
needs to be defined on the template. This can either be done by the user (one
time on one volume) or using some feature extraction policy. These points are
tracked through the entire sequence $T$ in order to determine the
correspondances and estimate the motion field $M_T ( \vec{x}, t )$. We solve
an energy minimization problem where we solve for the deformation vectors at
each of these points at each time frame. The energy function that is minimized
during this step is defined as:
\[ E = E_W + E_{\tmop{Smooth}} + E_{\tmop{diffeo}} \]
\begin{itemizearrow}
  \item Similarity: This is the wavelet similarity term. The similarity is
  computed between the wavelet attribute vectors at the point $M_T ( \vec{x},
  t )$, $W_T \text{$( M_T ( \vec{x}, t ) )$}$ and at the point $M_T ( \vec{x},
  t + 1 )$, $W_T ( M_T ( \vec{x}, t + 1 ) )$. The Energy function for
  similarity is defined as
  \[ E_W = \sum_{t = 0}^{N - 1} \sum_{\vec{x} \in \{ \vec{p} \}} \omega_{}_T (
     \vec{x}, t ) \left( \sum_{\vec{z} \in n ( \vec{x}, 0 )} \varphi \left(
     W_T ( M_T ( \vec{z}, t ) ), W_T ( M_T ( \vec{z}, t + 1 ) ) \right)
     \right) \]
  \item Smoothness: We wish to maintain a smooth path for all the points. We
  add a second order smoothing term to the motions obtained. This adds to the
  overall energy and defines the spatio-temporal smootheness of the motion
  field. The level of spatial and temporal smoothing is controlled by the
  parameters $\alpha$ and $\beta$. We define this energy as
  \[ E_{\tmop{Smooth}} = \alpha \cdot \sum_{t = 0}^{N - 1} \sum_{\vec{x} \in
     \{ \vec{p} \}} \| \frac{\partial^2 M_{_{} T} ( \vec{x}, t )}{\partial
     \vec{x}^2} \| + \beta \cdot \sum_{\vec{x} \in \{ \vec{p} \}} \sum_{t =
     0}^{N - 1} \| \frac{\partial^2 M_{_{} T} ( \vec{x}, t )}{\partial t^2} \|
  \]
  \item Spatial Diffeomorphic contraints: Impose diffeomorphic constraints by
  triangulating the original set of points and making sure the deformation is
  diffeomorphic. The energy function for the diffeomorphic constraint is
  computed by assigning a very high penalty for each vertex in the modified
  mesh that crosses over a facet in the tetrahedral mesh. Since we have other
  smoothness terms, this test is fast and sufficient to ensure that topology
  is not violated.
\end{itemizearrow}
We also impose constraint on the cyclic nature of the deformation. We impose a
hard constraint on the deformations to return a point $\vec{x}$ to itself
after the full cardiac cycle, i.e., $( \vec{x}, 0 ) = M_T ( \vec{x}, N )$.
This is done by ensuring that during each step of the minimization,
\[ \text{$\sum_{\vec{x} \in \{ \vec{p} \}} \| M_T ( \vec{x}, N ) - ( \vec{x},
   0 ) \| = 0$} \]

\subsubsection{Step 2: Combined estimation of 4D deformation field $H_{T
\leftarrow S} ( \vec{x}, t )$ and subject motion field $M_S ( \vec{x}$,t)}

The problem is again posed as one of minimizing an energy function. We define
the various components of our energy function. We estimate the 4D deformation
field and since the template motion field $M_T ( \vec{x}, t )$ is already
known at this stage, we get the subject motion field from these two
deformations. The energy function that our 4D registration algorithm minimizes
is defined as follows:
\[ E = E_F + E_B + E_C + E_{\tmop{SmoothM}} + E_{\tmop{SmoothH}} +
   E_{\tmop{diffeo}} \]
In order to make the registration independent of which of the two sequences
are treated as the template, the energy function should be symmetrical to the
two sequences being registered. Therefore we evaluate both the both the
forward transformation $H_{T \leftarrow S} ( \vec{x}, t )$ and the backward
transformation $H_{S \leftarrow T} ( \vec{x}, t ) = H_{T \leftarrow S}^{- 1} (
\vec{x}, t )$ and force them to be consistent with each other. These two
evaluations give rise to the forward and backward energy terms, $E_F$ and
$E_B$, respectively. These terms are defined as:
\begin{eqnarray*}
  &  & E_F = \sum_{t = 0}^{N - 1} \sum_{\vec{x} \in \Omega_T ( \vec{x} )}
  \gamma_{}_T ( \vec{x}, t ) \left( \sum_{( \vec{z}, \tau ) \in n ( \vec{x}, t
  )} d \left( G_T ( \vec{z}, \tau ), G_S ( H_{S \leftarrow T} ( \vec{z}, \tau
  ) ) \right) \right)
\end{eqnarray*}
\[ E_B = \sum_{t = 0}^{N - 1} \sum_{\vec{x} \in \Omega_S ( \vec{x} )}
   \gamma_{}_S ( \vec{x}, t ) \left( \sum_{( \vec{z}, \tau ) \in n ( \vec{x},
   t )} d \left( G_T ( H_{T \leftarrow S} ( \vec{z}, \tau ) ), G_S ( \vec{z},
   \tau ) \right) \right)  \]
We can obtain the subject motion field $M_S ( \vec{x}, t )$ in terms of the
template motion field $M_T ( \vec{x}, t )$  and the 4D deformation $H_{T
\leftarrow S} ( \vec{x}, t )$. This relation can be expressed as:
\[ M_S ( \vec{x}, t ) = H_{S \leftarrow T} ( M_T ( H_{T \leftarrow S} (
   \vec{x}, 0 ), t ) ) \]
The next term in the energy function is added because in addition to
maximizing the similarity between the subject and the template, we also need
that the correspondance that is implied on the subject as a result is also
correct. The third component in our energy function $E_C$ measures the
attribute vector matching of corresponding points in different time frames of
the subject $S ( \vec{x}, t )$. This term is similar to the wavelet similarity
term defined in Step 1. We can write this as
\[ E_C = \sum_{t = 0}^{N - 1} \sum_{\vec{x} \in \Omega_S ( \vec{x} )}
   \omega_{}_S ( \vec{x}, t ) \left( \sum_{\vec{z} \in n ( \vec{x}, 0 )}
   \varphi \left( W_S ( M_S ( \vec{z}, t ) ), W_S ( M_S ( \vec{z}, t + 1 ) )
   \right) \right) \]
The next two terms in the energy function are the smoothness terms. The
spatio-temporal smoothness term $E_{\tmop{SmoothM}}$ is similar to the
$E_{\tmop{Smooth}}$ term defined in Step 1, except that it is defined on the
subject. This is imposed to ensure that the estimated motion field on the
subject is spatio-temporally smooth. This can be written down as
\[ E_{\tmop{SmoothM}} = \alpha \cdot \sum_{t = 0}^{N - 1} \sum_{\vec{x} \in
   \Omega_S ( \vec{x} )} \| \frac{\partial^2 M_{_{} S} ( \vec{x}, t
   )}{\partial \vec{x}^2} \| + \beta \cdot \sum_{\vec{x} \in \Omega_S (
   \vec{x} )} \sum_{t = 0}^{N - 1} \| \frac{\partial^2 M_{_{} S} ( \vec{x}, t
   )}{\partial t^2} \| \]
The other smoothness term is the smoothness that needs to be enforced on the
4D deformation $H_{S \leftarrow T} ( \vec{x}, t )$. We add a second order term
here similar to the one added to smooth the motion field. This can be written
as
\[ E_{\tmop{SmoothH}} = \zeta \cdot \sum_{t = 0}^{N - 1} \sum_{\vec{x} \in \{
   \vec{p} \}} \| \frac{\partial^2 H_{S \leftarrow_{} T} ( \vec{x}, t
   )}{\partial \vec{x}^2} \| + \eta \cdot \sum_{\vec{x} \in \{ \vec{p} \}}
   \sum_{t = 0}^{N - 1} \| \frac{\partial^2 H_{S \leftarrow_{} T} ( \vec{x}, t
   )}{\partial t^2} \| \]
The diffeomorphic constraint is again calculated by ensuring that the
triangulated mesh transformed to the subject space does not violate topology.

Unlike Step 1 where evaluation of the wavelet similarity term was done in the
neighbourhood of focus points, we cannot guarantee that the subject also has
``{\tmem{good}}'' focus points at the corrsponding locations. Therefore, it
does not make sense to evaluate $E_C$ over the set of points $H_{T \leftarrow
S} ( \vec{x}, t )$. We instead take a random sample of points and evaluate
$E_C$ over these points. Since affine registration has been performed we can
selectively pick a distribution of points concentrated largely on the
myocardial area.

As in Step 1 we again define a hard constraint on the cyclic nature of the
deformation, this time on the subject. This can be written down as:
\[ \text{$\text{$\sum_{\vec{x} \in \Omega_S ( \vec{x} )} \| M_S ( \vec{x}, N )
   -_{} ( \vec{x}, 0 ) \| = 0$}$} \]

\subsubsection{Definitions}

\begin{description}
  \item[$\gamma_T ( \vec{x}, t )$] The weights for the geometric attribute
  vectors on the template.
  
  \item[$\gamma_S ( \vec{x}, t )$] The weights for the geometric attribute
  vectors on the subject.
  
  \item[$\omega_T ( \vec{x}, t )$] The weights for the wavelet attribute
  vectors on the template.
  
  \item[$\omega_S ( \vec{x}, t )$] The weights for the wavelet attribute
  vectors on the subject.
  
  \item[$n ( \vec{x}, t )$]   The neighbourhood about a point ($\vec{x}, t )$.
  
  \item[$\varphi ( \cdot, \cdot )$]   The similarity between two attribute
  vectors.
\end{description}

\end{document}
