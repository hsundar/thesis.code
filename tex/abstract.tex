Cardiac diseases claim more lives than any other disease in the world. Early diagnosis and
treatment can save many lives and reduce the associated socio-economic costs. Cardiac diseases
are characterized by both changes in the myocardial structure as well as changes in cardiac
function. Consequently, it is important for any cardiac diagnosis method to consider both these
aspects. Advances in MR Cine imaging methods have enabled us to acquire high-resolution 4D
images of the heart that capture the structural and functional characteristics of individual hearts.
However large inter and intra-observer variability in the interpretation of these images for the
diagnosis of diffuse cardiomyopathies has been reported \cite{bluemke03}. Studies also suggest that
standardizing acquisition protocols and objective analysis especially of regional myocardial
function will help improve the accuracy and reduce inter-observer variability \cite{pattynama93}. This has
created the need for sophisticated and highly automated image analysis methods, which can
identify and precisely quantify subtle and spatially complex patterns of structural and functional
changes in the heart. The development of such methods is the primary goal of this project.

Aim: Develop methods to measure myocardial function from MR Cine images, specifically
myocardial wall motion. We expect that the use of a model of cardiac motion to constrain the
motion estimation problem shall improve the accuracy of motion estimation.

Our hypotheses are that 1) these methods will improve the accuracy and robustness of estimating
myocardial motion from MR Cine sequences (Aim 1), 2) these methods will detect subtle and
spatio-temporally distributed structural and functional differences between patients and healthy
individuals, thereby enabling the construction of sensitive and specific diagnostic tools for the
early detection of ARVC and other diffuse cardiomyopathies that currently remain undetectable,
especially at early stages (Aims 2 and 3).

Although the algorithms and methods developed as part of this work should apply in general to
the whole class of diffuse cardiomyopathies, we restrict the scope to the characterization of
Arrhythmogenic right ventricular cardiomyopathy (ARVC). Future work shall focus on how the
methods developed as part of this work can be generalized to other diffuse cardiomyopathies.
