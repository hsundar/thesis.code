\chapter{Introduction}
\label{intro}
 
\section{Motivation} 
According to WHO estimates, 16.7 million people around the world die of cardiovascular diseases (CVD) each year \cite{aha}. Of the total CVD deaths annually, about 8.6 million are of women. Heart attack and stroke deaths are responsible for twice as many deaths in women as all cancers combined. The US health care system is facing serious access and quality issues. The current access and quality issues will be compounded in the coming years by three factors that will serve to accelerate the rate of cardiovascular disease and its complications. 

First, aging of the population will undoubtedly result in a concomitant increase in the incidence of chronic diseases, including coronary artery disease, heart failure and stroke. Second, we are experiencing an explosive increase in the prevalence of obesity and type 2 diabetes and their related complications of hypertension, hyperlipidemia, and artherosclerotic vascular disease \cite{aha}. Finally, there is an alarming increase in unattended risk factors in the younger generations that will continue to fuel the cardiovascular epidemic for years to come. These factors include obesity and smoking. CVD is the leading cause of mortality in every region in the world except sub-Saharan Africa, and it is anticipated that cardiovascular disease will eclipse the present the current leader in that region, infectious disease, within the next few years. By 2020 the WHO estimates nearly 25 million CVD deaths worldwide. By 2020, cardiovascular diseases, injury and mental illnesses will be responsible for about one half of all deaths  and one half of all healthy years lost, worldwide. The socio-economic impact of cardiovascular disease is too great to be ignored. In 2004 the estimated direct and indirect cost of CVD was \$368.4 billion \cite{aha}.

It is noteworthy that the principal cardiovascular disorder responsible for the global rise in mortality is no longer rheumatic heart disease, but rather artherosclerotic vascular disease. Ischemic heart disease is the leading cause of death in the world, and cerebrovascular disease is the second leading cause \cite{aha}. It is often assumed that artherosclerosis is a disease of the affluent, industrialized countries. However, 80\% of these deaths occur in low-to-middle income countries of varying size like China, Russia, Poland, Mauritius, Argentina, and India \cite{wha}. In many countries, the need for care already outstrips the ability to provide it to its citizens. Throughout the world, even in economically advanced societies, there are deficiencies in preventive and acute care that might stem the tide of this epidemic. Because of these reasons, it is important to stop think of cardiovascular diseases as a rich-man's disease and start dealing with it as an epidemic. Consequently, it is extremely important to develop cheap, non-invasive early detection methods to be able to identify the onset of cardiovascular diseases and treat them before they cause excessive damage. 

There are a number of invasive and non-invasive procedures that are currently used for diagnosis of CVD \cite{merck}. To reduce trauma for the patient it is preferable to use non-invasive procedures. Important noninvasive techniques are plain radiography, radionucleotide imaging, positron emission tomography (PET), Magnetic Resonance Imaging (MRI) and Ultrasound \cite{merck}. Of these, MRI can provide much cardiac information during a single examination and may thus be more cost-effective than several other studies. In the last few years MRIs have become cheaper and more accessible and given the human and economic cost of CVD, it is important that high-risk populations be screened regularly for CVD. 

Cardiac diseases are characterized by both changes in the myocardial structure as well as changes in cardiac function. Consequently, it is important for any cardiac diagnosis method to consider both these aspects. Advances in MR Cine imaging methods have enabled us to acquire high-resolution 4D images of the heart that capture the structural and functional characteristics of individual hearts. However large inter and intra-observer variability in the interpretation of these images for the diagnosis of diffuse cardiomyopathies has been reported \cite{bluemke03}. Studies also suggest that standardizing acquisition protocols and objective analysis especially of regional myocardial function will help improve the accuracy and reduce inter-observer variability \cite{pattynama93}. This has created the need for sophisticated and highly automated image analysis methods, which can identify and precisely quantify subtle and spatially complex patterns of structural and functional changes in the heart. The main contribution of this thesis is the development of computational methods to characterize myocardial function from MR images.

Although the algorithms and methods developed as part of this work should apply in general to the whole class of diffuse cardiomyopathies, we restrict the scope to the characterization of Arrhythmogenic right ventricular cardiomyopathy (ARVC). Future work shall focus on how the methods developed as part of this work can be generalized to other diffuse cardiomyopathies.

%Therefore, MRI being a simple non-invasive method is an obvious choice for a screening test for CVDs. However, for this to happen in an effective manner it is important to develop diagnostic techniques that can quickly and effectively screen patients, making it faster and more cost effective compared to the current method of having specialists process all patients. 
%
%It takes an expert radiologist 20 minutes, on an average,to evaluate an MR scan of a patient for signs of CVD. This is generally the case for localized abnormalities like ischaemia. The diagnosis time can run into hours for non-localized CVDs like Arrhythmogenic right ventricular cardiomyopathy (ARVC) \cite{rvd}. Because of the sheer population that is at risk, it is important that these detection method's be automated, since detailed diagnosis by human experts is next to impossible. Therefore it is very important to develop Computer aided diagnosis (CAD) algorithms for cardiac diseases and have the experts further analyze the patients screened by the CAD algorithm for specific diagnosis and treatment. Even in these cases, the CAD algorithm can highlight abnormal behavior and make the job of the radiologist much easier. Developing the tools and algorithms for such a diagnosis method is the main goal of this work.

\section{Arrhythmogenic Right Ventricular Cardiomyopathy}

\subsection{Clinical Features and Relevance}

Arrhythmogenic right ventricular cardiomyopathy is generally accepted as the most common cause of sudden cardiac death in young patients, and despite over 25 years of study remains a poorly understood disease \cite{ferrari2003arv, thiene1988rvc}. Recent genetic studies have elucidated both autosomal dominant and recessive inheritance mechanisms \cite{paul2003gar}. Arrhythmogenic right ventricular cardiomyopathy (ARVC), also known as arrhythmogenic right ventricular dysplasia is characterized by progressive fibrofatty replacement of right ventricular myocardium, initially with typical regional and later global right and some left ventricular involvement, with relative sparing of the septum \cite{thiene1988rvc}. As the underlying pathophysiology of ARVC remains unknown, there is no consensus regarding a gold standard for diagnosis \cite{thiene2000pap}. Diagnosis of ARVC is based on presence of major and minor criteria that include structural, histological, electrocardiographic, arrhythmic, and genetic factors. At its early stages, the diagnosis of ARVC remains a clinical challenge. The different imaging modalities play a limited role, as there is no single non- invasive method that helps to establish or exclude this diagnosis.

\subsection{MR Imaging of ARVC}
While Magnetic Resonance imaging (MRI) findings were not included in the original Task Force criteria because of a lack of evidence of efficacy, it has been evaluated as a method for demonstration of the structural and functional abnormalities listed, including myocardial fatty replacement, RV dilation, wall thinning, and aneurysm formation, and evaluation of RV function \cite{bluemke03}. However, the diagnostic performance for detection of these abnormalities, as measured by sensitivity, specificity, and inter-observer variability remains somewhat uncertain. A recent single center study compared MR findings in 12 patients with definitive diagnosis of ARVC by Task Force criteria, with 10 age and sex matched controls \cite{tandri2003mri}. The study found evidence of intramyocardial fat in 75\% of the patients and none of the controls, a greater incidence of RV hypertrophy and statistically significant increases in quantitative measures of RV dimensions and decreases in RV function. In another study \cite{bluemke03}, 13 readers at multiple institutions reviewed images from MR evaluations of 7 patients with a diagnosis of ARVC by Task Force criteria, 6 controls, and 32 patients with suspected ARVC. While the presence of reported RV enlargement and other morphological abnormalities were significantly higher in the definitive ARVC patients, the percentages with reported intramyocardial fat were equal amongst the three groups. Overall diagnostic quality was poor and there was wide inter-observer variability for all parameters evaluated. Limitations of this study included non-standardization of MR acquisition techniques and criteria for interpretation, as well as lack of inclusion of functional cine images in the interpretations. The findings suggest that standardizing acquisition protocols and standardized, objective analysis especially of regional myocardial function will improve the accuracy and reduce variability.

\section{Assessing Myocardial Function}

Cardiomyopathies present themselves in different forms, both by structural changes like plaque formation, fat deposits, etc., and functional changes like variations in ventricular wall motion, ejection fraction, and perfusion. Both of these need to be extracted from the image before accurate diagnosis can be done. A lot of work has been done in feature extractors for structural characterizations of disease. This has focused primarily on extracting image features like intensities and gradients \cite{ intgrad}, moments \cite{ hammer,  moments}, Gabor features \cite{ manju96}, and local frequency representations \cite{ locfreq}. The problem with cardiomyopathies is that not all of them can be characterized by structural changes. Function at rest may be abnormal as a result of one of the spectrum of ischemic heart diseases (ischemia, infarction, hibernation) or of cardiomyopathy from other causes. During stress testing, new or worsening wall motion abnormalities are indicative of functionally significant coronary artery stenosis \cite{smart2000das}. In addition, wall motion imaging to detect regional contractile reserve is an accurate measure of myocardial viability, and the results can help guide coronary revascularization therapy. Characterizing cardiomyopathies based on both structural and functional changes will make the diagnosis algorithm more accurate and robust. Of course quantization of the myocardial wall motion represents a challenge in itself. 

Most clinical modalities used to image myocardial function evaluate passive wall motion (ventriculography) or wall thickening (echocardiography, gated single-photon emission computed tomography, or cine MR imaging). MR imaging also allows quantitative measurement of regional intramyocardial motion and, subsequently, strain, which can be more sensitive to wall motion abnormalities than is wall thickening. MR imaging methods for the quantification of intramyocardial wall motion can be loosely classified into two approaches, those relying on specially developed MR imaging protocols to help in the estimation of myocardial motion and those relying on image analysis techniques to extract motion estimates from MR Cine sequences.

\subsection{Specialized MR Protocols}
\subsubsection{MR Tagging}
MR Tagging was developed to provide non-invasive virtual markers inside the myocardium, which deform with myocardial motion \cite{ mrtag}. MR imaging and especially tagged MR are currently the reference modalities to estimate dense cardiac displacement fields with high spatial resolution. The deformation fields, as well as the derived motion parameters such as myocardial strain can be determined within an accuracy of 2mm x 2mm \cite{ Shi99, chenBook}. The primary disadvantage of tagging is the reduced spatial resolution of strain relative to the image spatial resolution. In tagging, after the displaced tag lines are detected \cite{ guttman94}, the displacement field can be estimated and intramyocardial strain can be computed in a variety of ways \cite{ mrtag}. With this approach, although strain may be interpolated to any desired spatial resolution, the fundamental spatial resolution of strain is nominally determined by the distance between the tag lines, which is typically several pixels. Tag detection has an additional disadvantage in that it typically requires substantial manual intervention and is therefore a time-consuming task. Harmonic phase analysis will likely obviate tag detection \cite{osman1999cmt}, but the spatial resolution of the resultant strain maps will not necessarily improve. The spatial resolution of strain maps obtained from tagged images after harmonic phase analysis is determined by the k-space filter of the analysis; in practice with single breath-hold acquisitions, the resolution has been relatively poor \cite{ garot00}. Additionally since the right ventricle (RV) is much thinner than the left ventricle (LV), it is difficult to place more than a single tag within the RV, making the estimation of RV motion extremely difficult and inaccurate. Since we are most interested in the characterizing RV function, tagging is not appropriate for our purpose.

\subsubsection{Phase Contrast Imaging}
The second approach is that of MR phase contrast imaging \cite{ mrphase}, which is based on the concept that spins that are moving in the same direction as a magnetic field gradient develop a phase shift that is proportional to the velocity of the spins. This information can be used directly to determine the velocity of the spins, or in the cardiac case the velocity of any point within the myocardium. The main problem with this approach is that four acquisitions have to be made for each heart, one the regular MR cine sequence and one phase contrast acquisition each for the velocity components in the x, y, and the z directions. Consequently, MR phase contrast imaging is not used much in a clinical setting. 

\subsubsection{DENSE and HARP}
Displacement-encoded imaging with stimulated echoes (DENSE) \cite{epstein2004dec} and harmonic phase imaging (HARP) \cite{osman1999cmt} employ 1-1 spatial modulation of magnetization to cosine modulate the longitudinal magnetization as a function of position at end diastole. Later in the cardiac cycle the cosine-modulated signal is sampled and used to compute myocardial strain from the signal phase. The sampled signal generally includes three distinct echoes:  a displacement-encoded stimulated echo, the complex conjugate of the displacement-encoded echo, and an echo arising from T1 relaxation. If the T1-relaxation and complex conjugate echoes are suppressed, then a phase image representing just the displacement-encoded echo can be reconstructed. However, data-acquisition in single-breath-hold DENSE MR imaging has been limited to only one cardiac phase. Multiple breath-hold DENSE produces images at multiple phases of the cardiac cycle, but the resolution has been poor and is fundamentally a 2D approach and the estimation of through-plane displacement has been poor. 

\subsection{Extracting motion from MR Cine images}
An alternate approach is to estimate myocardial motion from MR Cine sequences. MR Cine images in a clinical setting at sub millimeter resolutions (in-plane), with slice thickness in the range of 6-10mm. The temporal resolution varies between 25-70ms. A lot of work has been done in extracting cardiac motion fields from MR and Ultrasound image sequences \cite{ Shi99,  ledesma01, McE00, Papa01, perperidis04,  Song91,  Wang01}. These can be classified into two main categories. The first approach uses segmentation of the myocardial wall, followed by geometric and mechanical modeling using active contours or surfaces to extract the displacement field and to perform the motion analysis \cite{ Shi99,  Papa01,  Wang01}. For matching two contours or surfaces, curvatures are frequently used to establish initial sparse correspondences, followed by the dense correspondence interpolation in other myocardial positions by regularization or mechanical modeling \cite{ Shi99, McE00}. The lack of distinct landmarks on the myocardial wall makes it difficult to estimate the wall motion based on surface tracking. In addition this approach is very sensitive to the accuracy with which the myocardium can be segmented. Also it performs poorly in regions within the myocardium, and manages to only align the myocardial boundaries. The other approach uses energy-based warping or optical flow techniques to compute the displacement of the myocardium \cite{ ledesma01,  perperidis04,  Song91}. Perperidis et al. \cite{ perperidis04} use a regular grid with a B-spline basis to model the deformation and use normalized mutual information as the similarity metric which is calculated over the whole image. One of the major shortcomings of these approaches is that the transformation estimated as a result of the registration is not unique and in fact does not necessarily conform to the underlying myocardial motion. The same algorithm can give different estimates of motion for different initial guesses and different parameters. The problem arises since these methods attempt to maximize the image similarity with only a smoothness constraint on the transformation. Since there can be many transformation that can map an image onto another (especially sparsely sampled ones as in the case of MR Cines) there is no guarantee that the estimated transformation is the correct one \cite{ Cachier:MICCAI:01,  Cachier-JMIV-2004}. These methods estimate motion by evolving the current estimate of motion under the action of external image forces (image similarity) and internal forces which constrain the regularity of the motion (smoothness). Such regularizers work well with respect to noise removal but they do not incorporate a priori knowledge of the underlying cardiac motion. Incorporating a biomechanically-inspired model for the myocardium has the potential for a more accurate motion estimation \cite{mcculloch1998cbh}. Functional models of the heart are direct computational models, designed to reproduce in a realistic manner the cardiac activity, often requiring high computational costs and the manual tuning of a very large set of parameters. Such methods can be computationally prohibitive for our purposes, and we instead select a level of modeling compatible with reasonable computing times and involving a limited number of parameters. Such simplifications add additional modeling errors, but our hypothesis is that in spite of these modeling errors the estimated motion fields shall be more accurate than those obtained from approaches not incorporating a priori knowledge. A detailed and thorough review of cardiac image registration methods can be found in \cite{ makela02} and a general review of image registration methods can be found in \cite{ Zitova03}. 

\section{Biomechanical Modeling of the Heart}

\section{Contributions}

\section{Organization of this Thesis}