\section{Delaunay triangulation}

The Delaunay triangulation of a point set is a collection of edges satisfying an "empty circle" property: for each edge we can find a circle containing the edge's endpoints but not containing any other points. The Delaunay triangulation is the dual structure of the Voronoi diagram in R�. By dual, we mean to draw a line segment between two Voronoi vertices if their Voronoi polygons have a common edge, or in more mathematical terminology: there is a natural bijection between the two which reverses the face inclusions.

The circumcircle of a Delaunay triangle is called a Delaunay circle.